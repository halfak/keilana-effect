When I first saw this gap-to-surplus shift, I honestly had no idea what could have caused it.  In an effort to share my analysis with a larger community, I presented the preliminary results in a prominent public forum for Wikimedia-related research projects -- the Wikimedia Research Showcase\footnote{\url{https://www.mediawiki.org/wiki/Wikimedia_Research/Showcase}}.  During the presentation, Wikipedia editors who ran a series of outreach initiatives to bring attention to biographies about Women Scientists reached out to us to let us know that the beginning of their initiatives corresponded to the beginning of the period of the shift.  User:Keilana (aka Emily Temple-Wood) and her collaborators received substantial attention from the media for their work to increase coverage of Women Scientists in Wikipedia~\cite{change13emily}.  At the most basic level, my analysis of content quality and coverage of this topic in Wikipedia seems to suggest that their work has had a very large effect.

I haven't been able to determine what might explain the unusual shape of the trajectory for the proportion of B-class articles.  One hypothesis is related to the process by which GA and FA-class articles are assessed.  Unlike the lower quality classes, GA and FA-class articles go through a formal peer review process and are promoted in other places in the wiki.  It's possible that there's a large incentive to bring any B-class article to the next stage (GA) since it's only one step away and it is rewarded with public recognition.

The smaller proportion of articles that are predicted to be at the FA-class level is maybe a bit more concerning for the coverage of content about Women Scientists.  One possibility is that the model is biased against articles about Women Scientists.  There's a well discussed bias against coverage of Women Scientists in the type of reference work that Wikipedia tends to cite.  On one had, it could be that articles about Women Scientists are inherently shorter and just look to be lower quality that other articles in the wiki.  It could also be that it's hard to write a truly comprehensive article about a Woman Scientist for the same reason.  It turns out that Wikipedians' own assessments suggest that there's a similar proportion of FA-class articles about Women scientists(7/5681 = 0.12\%~\cite{womenscientist_stats}) as there are across the entire wiki (6k/5m = 0.12\%~\cite{wikipedia1.0stats}), so it could be possible that the model is showing a slight bias here.
