Since it's inception, quality has been the most prominent concern with regards to the future of Wikipedia. After all, how can high quality information artifacts be produced when there's literally no restriction on who is allowed to contribute? Over the past 12 years (as of 2017), the research literature around Wikipedia has advanced our understanding of the open encyclopedia's quality and the processes by which crowds of volunteers can manage such an information artifact.

Our first major leaps in understanding of Wikipedia's quality dynamics happened around the time that Jim Giles published a report in nature (2005)\cite{giles05internet} that surprised the world. This seminal report showed that Wikipedia's coverage of scientific content compared favorably (and in some ways, better) than the more dominant encyclopedias that were used. Since that surprising result was published, researchers have been pushing toward greater understanding of how open, volunteer processes could have generated such a high quality information resource.

While we do know a lot about quality dynamics in Wikipedia, there are still many questions that remain. Where are Wikipedia's coverage gaps? What types of editing patterns lead to efficient quality improvements? These questions are important for the science and for the practice of Wikipedians. In this paper, we'll detail the development of a measurement strategy and the release of a public dataset that I believe will make answering these questions far easier than ever before. In the following sections, I'll summarize the state of the art with regards to quality dynamics and measurement in Wikipedia, I'll explain my measurement methodology, and I'll provide a demonstration analysis that
gives novel insights into the coverage gaps and quality dynamics of articles about Women Scientists.
