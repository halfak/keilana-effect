As in past work that used Wikipedia's quality assessments as an outcome measure, we're assuming that Wikipedians' notions of \emph{quality} correspond to some general, fundamental true quality.  But quality is a more complex concept that is arguably context and purpose dependent.  For some users and uses of Wikipedia, a Stub-class may be perfectly acceptable (\emph{e.g.} settling a bar bet), while for others an FA-class would be necessary to really get a complete overview of the subject.  I have decided to model Wikipedians' own quality ratings because their context and purpose is the construction of an encyclopedia.  While this may not fit all contexts and purposes, it does seem relevant to a substantial cross-section of the research literature that is concerned about the efficiency of open production processes and it is clearly useful for ORES' intended audience: Wikipedians.

As was mentioned in section~\ref{sec:modeling_actionable_quality}, measuring the \emph{completeness} of a cross-section of Wikipedia is difficult when it's not clear how many articles the cross-section \emph{should} have.  We've chosen to operate on the assumption that the number of articles present in the cross section at the time of measurement is a useful denominator to use historically.  Future work could explore the development of a more reasonable denominator by taking advantage of indexes of known notable topics that might eventually have an article in Wikipedia.  Notably, User:Emijrp has already put substantial effort into just such an initiative~\cite{emijrp17all}.  He estimates that the current total number of articles in English Wikipedia (about 5 million) represents only about 5\% of all of the articles that will eventually be covered in the encyclopedia.
