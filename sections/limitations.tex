As in past work that used Wikipedia's quality assessments as an outcome measure, we're assuming that Wikipedians' notions of \emph{quality} correspond to some general, fundamental \emph{true quality}.  But quality is a more complex concept that is arguably context and purpose dependent.  For some users and uses of Wikipedia, a Stub-class may be perfectly acceptable (\emph{e.g.} settling a bar bet), while for others an FA-class would be necessary to really get a complete overview of the subject.  I have decided to model Wikipedians' own quality ratings because their context and purpose is the construction of an encyclopedia.  While this may not fit all contexts and purposes, it does seem relevant to a substantial cross section of the research literature that is concerned about the efficiency of open production processes and it is clearly useful for ORES' intended audience: Wikipedians.

Another concern about his method is the \emph{meaning} behind changes in the predicted quality level of an article over time.  As discussed in the \nameref{sec:methods_for_measuring} section, the article quality prediction model was formally evaluated against withheld data (wikipedians' assessments) and these statistics suggest a high level of fitness.  However, Wikipedians direct their own assessment activities and that means there may be something special about the revisions that are generally assessed.  Essentially, there are many unassessed versions of articles between the revisions that were directly assessed by Wikipedians and we don't have a ground truth about the quality of those revisions to compare against.  One of the primary claims that I make in this paper is that the article quality model deployed in ORES (that was used to generate the linked dataset and perform the demonstration includedin this paper) is an effective means to assess these otherwise unassessed versions of article content.  I justify this conclusion by (informally) observing that that the temporal dynamics of article quality as measured by the model seem to closely reflect reality.  I also note that ORES' users and I have used the model \emph{in practice}.  It is from this that I conclude that the temporal dynamics of predicted quality represent useful information about real changes in article quality.  While this type of assessment may be reasonable for the purposes of this study, a formal analysis of the prediction model's ability to predict the quality inbetween natural assessments is desirable.  Future work could ask Wikipedians to directly assess a random set of otherwise unassessed revisions and measure the fitness of the model against those revisions in to perform such a formal analysis.

As was mentioned in the \nameref{sec:modeling_actionable_quality} section, measuring the \emph{completeness} of a cross section of Wikipedia is difficult when it's not clear how many articles the cross section \emph{should} have.  I have chosen to operate on the assumption that the number of articles present in the cross section at the time of measurement is a useful denominator to use historically.  Future work could explore the development of a more reasonable denominator by taking advantage of indexes of known notable topics that might eventually have an article in Wikipedia.  Notably, User:Emijrp has already put substantial effort into just such an initiative~\cite{emijrp17all}.  He estimates that the current total number of articles in English Wikipedia (about 5 million) represents only about 5\% of all of the articles that will eventually be covered in the encyclopedia.

Finally, it's important to note that this paper does not do a study that would be able to rigorously conclude the direct \emph{causal relationship} between Keilana's initiatives and the coverage of women scientists in Wikipedia.  Such a conclusion may be apparent, but there are other potential explanations for the observed correlation.  For example, it could be that there was generally a sudden surge in interest around women scienists in the beginning of 2013 from which both Keilana's initiatives and the quality of articles about women scientists were independent effects.  Future work may examine this by analysing the contributions in this content space and qualitatively studying the motivations of these volunteers to find out if the initiatives or something else had inspired them to do take on this work.
