\begin{table}
\centering
\caption{Wikipedia 1.0 (wp10) assessment rating scale. The 6 levels of assessment are provided with the ``readers experience'' description copied from Wikipedia\protect\footnote{\url{https://enwp.org/Wikipedia:WikiProject_assessment}}.  Each level also has a well-defined, detailed set of assessment criteria that involves discussion of formatting, coverage, and proper sourcing.}
\begin{tabular}{|c|p{7cm}|} \hline
& \multicolumn{1}{|c|}{summary} \\
\hline
FA & Professional, outstanding, and thorough; a definitive source for
encyclopedic information.\\ \hline
GA & Useful to nearly all readers, with no obvious problems; approaching
(but not equalling) the quality of a professional
encyclopedia.\\ \hline
B & Readers are not left wanting, although the content may not be
complete enough to satisfy a serious student or
researcher.\\ \hline
C & Useful to a casual reader, but would not provide a complete picture
for even a moderately detailed study.\\ \hline
Start & Provides some meaningful content, but most readers will need
more.\\ \hline
Stub & Provides very little meaningful content; may be little more than
a dictionary definition.\\ \hline
\end{tabular}
\label{tab:assessment_scale}
\end{table}


\leadin{Wikipedia assessment ratings.} Wikipedians assign quality assessments to articles based on a scale that was originally developed to produce an official "1.0" version of Wikipedia\footnote{\url{https://en.wikipedia.org/wiki/Wikipedia:Version_1.0_Editorial_Team}}.  This scale has since been adopted by WikiProjects\footnote{\url{https://en.wikipedia.org/wiki/WikiProject}}, self-organized subject-matter focused working groups on Wikipedia (\emph{e.g.} WikiProject Video Games, WikiProject Medicine, and WikiProject Breakfast).  Table~\ref{tab:assessment_scale} shows the rating scale with defunct old grades ("A", "B+", etc.) removed.  Wikipedians use this scale to track progress towards content coverage goals and to build work lists (\emph{e.g.} WikiProject Medicine's tasks\footnote{\url{https://en.wikipedia.org/wiki/Wikipedia:WikiProject_Medicine/Tools\#Tasks}}).

Both Kittur et al.\cite{kittur08harnessing} and Arazy \& Nov\cite{arazy10determinants} used article quality assessments provided by Wikipedians to track article improvements and to find correlations with editor activity and experience level characteristics that they use to draw conclusions about successful collaboration patterns.  Regretfully, the process by which Wikipedians assess and re-assess articles is unpredictable.  We can be relatively sure of the quality level of an article at the time it was assessed, but there's no clear way to know when, exactly, the quality level of an article actually changed.  The aforementioned studies use a set of complex propensity modeling strategies like Heckman Correction\footnote{\url{https://en.wikipedia.org/wiki/Heckman_correction}} to minimize the possibility that the correlations they observe were simply due to the \emph{assessment} behavior of Wikipedians and not actual changes in article quality.  The modeling of correction is difficult to optimize and evaluate independently of the effects that editor collaboration patterns may have had on those quality changes.  Further, past analyses have been limited to the times at which Wikipedians were performing assessments of articles.  Assessments didn't become common until 2006 (5 years after Wikipedia's inception) and the criteria by which articles are assessed has been undergoing changes since then.  Articles that were "B" class in 2006 would likely now be classified as a "Start" class due to insufficient inline references.  Table~\ref{tab:b_class_criteria} shows the change in B-class criteria between 2006 and 2017.

Further, the fact that assessments are sparse and unpredictable also means that the assessment are often out of date.  This is an operational issue for Wikipedians too.  WikiProject groups organize re-assessment drives to bring the assessments of articles under their purview into compliance, but this is a never-ending process.  Since Wikipedia contributions come from anyone (member of the WikiProject or not) and re-assessments are uncommon despite regular coordinated efforts, the overall assessments of Wikipedia articles are perpetually out of date and reassessments are a never-ending source of new work.

\begin{table*}
\centering
\caption{B-class criteria comparison. The B-class criteria is compared between December of 2006\protect\footnote{\url{https://en.wikipedia.org/wiki/?oldid=96568092}} and April of 2017\protect\footnote{\url{https://en.wikipedia.org/wiki/?oldid=769814730}}.  Note the increase in detail generally and the discussion of the importance of inline citations specifically.  Also note that C-class didn't exist as of December 2006.}
\begin{tabular}{|c|p{16.4cm}|} \hline
2006
& Has several of the elements described in ``start'', usually a majority of the material needed for a completed article. Nonetheless, it has significant gaps or missing elements or references, needs substantial editing for English language usage and/or clarity, balance of content, or contains other policy problems such as copyright, Neutral Point Of View (NPOV) or No Original Research (NOR). With NPOV a well written B-class may correspond to the ``Wikipedia 0.5'' or ``usable'' standard. Articles that are close to GA status but don't meet the Good article criteria should be B- or Start-class articles. \\ \hline
2017
&
\begin{tabular}{p{16cm}}
1.~The article is suitably referenced, with inline citations. It has reliable sources, and any important or controversial material which is likely to be challenged is cited. Any format of inline citation is acceptable: the use of \texttt{<ref>} tags and citation templates such as \texttt{\string{\string{cite web\string}\string}} is optional. \\ \vspace{.5ex}
2.~The article reasonably covers the topic, and does not contain obvious omissions or inaccuracies. It contains a large proportion of the material necessary for an A-Class article, although some sections may need expansion, and some less important topics may be missing. \\ \vspace{.5ex}
3.~The article has a defined structure. Content should be organized into groups of related material, including a lead section and all the sections that can reasonably be included in an article of its kind. \\ \vspace{.5ex}
4.~The article is reasonably well-written. The prose contains no major grammatical errors and flows sensibly, but it does not need to be ``brilliant''. The Manual of Style does not need to be followed rigorously.
5.~The article contains supporting materials where appropriate. Illustrations are encouraged, though not required. Diagrams and an infobox etc. should be included where they are relevant and useful to the content. \\ \vspace{.5ex}
6.~The article presents its content in an appropriately understandable way. It is written with as broad an audience in mind as possible. Although Wikipedia is more than just a general encyclopedia, the article should not assume unnecessary technical background and technical terms should be explained or avoided where possible.
\end{tabular} \\ \hline
\end{tabular}
\label{tab:b_class_criteria}
\end{table*}

