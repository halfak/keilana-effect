But how did Wikipedia arrive at such a high quality level?  There are two major threads of research in this area: counter-vandalism and article quality dynamics.

\leadin{Counter-vandalism.} Due to it's open nature, Wikipedia is under constant threat of vandalism and other types of damaging changes.  At it's most basic level, Wikipedia protects against damage by maintaining a history of all versions of every article.  This allows ``patrollers'' to easily clean up damage whenever it is discovered.  A \emph{revert}\footnote{\url{https://meta.wikimedia.org/wiki/Research:Revert}} is a special type of edit that removes the changes of another edit -- usually by restoring the last good version of the article.

But reverts are not the whole story of counter-vandalism in Wikipedia.  There is complex network of communities of practices (like the conter-vandalism unit\footnote{\url{https://en.wikipedia.org/wiki/Wikipedia:CVU}}), policies (like the 3-revert-rule\footnote{\url{https://en.wikipedia.org/wiki/Wikipedia:3RR}}), and highly advanced automated tools that support editors in finding and quickly removing damaging contributions to Wikipedia\cite{geiger10work}\cite{priedhorsky07creating}.  Together this socio-technical system fills the infrastructural role of keeping out the bad stuff -- ensuring that Wikipedia's open nature does not cause quality to decay into nonsense.

\leadin{Article quality dynamics.} While counter-vandalism and other types of patrolling work helps keep the bad out of the Wiki, there are other dynamics and processes at play that determine which articles will increase in quality efficiently.  Stvilia et al. first hypothesized a framework for how high quality content was generated in Wikipedia\cite{stvilia08information}.   Essentially, volunteers will allow their interests to drive where they contribute, and through building on to each others' work, articles grow and are gradually refined.  This interest-driven pattern could likely explain how Wikipedia's demographic gaps have lead to coverage and quality gaps\cite{lam11clubhouse}.

Through Kittur et al.'s work, we know that group structure and dynamics play an important role in the efficiency of article improvement.  They showed that articles where few editors make most of the edits (and the vast majority of editors contribute very little individually) tend to increase in quality more quickly than articles where the distribution of edits per editor is more uniform\cite{kittur08harnessing}.  Arazy \& Nov extend this conclusion by showing that certain types of editor experience are critical to article improvement -- that it is not only important to have diversity in contribution rates but also diversity in the total experience level of Wikipedia editors\cite{arazy10determinants}.  In order to draw these conclusions, the authors of both of these studies needed to operationalize measurements of quality in Wikipedia and compare the configurations of editors and their contribution types to changes in measured quality.
