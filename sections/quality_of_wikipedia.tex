Giles' study set in motion a series of studies examining the quality of several different subject spaces in the encyclopedia.  Mesgari et al. provides an excellent survey of this work\cite{mesgari15sum}, but for the purposes of this paper, I'll summarize their key findings as follows:

\begin{itemize}
\item Wikipedia's coverage is broad and comprehensive.
\item Wikipedia has a high level of currency -- especially with regards to topics of interest to the public.
\item Wikipedia's accuracy compares favorably with traditional encyclopedias.
\end{itemize}

Yet the story isn't all one of pure success.  Given the conclusions drawn from years of research, Wikipedia's coverage of most major topics that are covered in traditionally encyclopedias is hard to question.  However, not all topics are covered as completely as others.  Gaps in Wikipedia's content coverage are concerning because of the encyclopedia's dominance as an information resource.  600 million people read the encyclopedia every month\footnote{\url{https://tools.wmflabs.org/siteviews/?platform=all-sites&source=unique-devices&start=2016-04&end=2017-03&sites=en.wikipedia.org}}.  With an information source of such popular use and apparent completeness, any topic that is covered less completely than others (or not at all) could imply less importance or relevance to certain types of knowledges.  Based on past work, we know that Wikipedia lacks in coverage of topics about Women~\cite{reagle11gender}, of interest to Women~\cite{lam11clubhouse}, about rural and non-Western geographies~\cite{johnson16home,hecht09measuring} and cultures~\cite{graham14uneven}.  Thus, understanding where Wikipedia's coverage is lacking and which initiatives are effective in closing coverage gaps is critical to the long term success of the project and is a concern for the preservation of \emph{all} human knowledge.
