\begin{table*}
\centering
\caption{B-class criteria comparison. The B-class criteria is compared between December of 2006\protect\footnote{\url{https://en.wikipedia.org/wiki/?oldid=96568092}} and April of 2017\protect\footnote{\url{https://en.wikipedia.org/wiki/?oldid=769814730}}.  Note the increase in detail generally and the discussion of the importance of inline citations specifically.  Also note that C-class didn't exist as of December 2006.}
\begin{tabular}{|c|p{16.4cm}|} \hline
2006
& Has several of the elements described in ``start'', usually a majority of the material needed for a completed article. Nonetheless, it has significant gaps or missing elements or references, needs substantial editing for English language usage and/or clarity, balance of content, or contains other policy problems such as copyright, Neutral Point Of View (NPOV) or No Original Research (NOR). With NPOV a well written B-class may correspond to the ``Wikipedia 0.5'' or ``usable'' standard. Articles that are close to GA status but don't meet the Good article criteria should be B- or Start-class articles. \\ \hline
2017
&
\begin{tabular}{p{16cm}}
1.~The article is suitably referenced, with inline citations. It has reliable sources, and any important or controversial material which is likely to be challenged is cited. Any format of inline citation is acceptable: the use of \texttt{<ref>} tags and citation templates such as \texttt{\string{\string{cite web\string}\string}} is optional. \\ \vspace{.5ex}
2.~The article reasonably covers the topic, and does not contain obvious omissions or inaccuracies. It contains a large proportion of the material necessary for an A-Class article, although some sections may need expansion, and some less important topics may be missing. \\ \vspace{.5ex}
3.~The article has a defined structure. Content should be organized into groups of related material, including a lead section and all the sections that can reasonably be included in an article of its kind. \\ \vspace{.5ex}
4.~The article is reasonably well-written. The prose contains no major grammatical errors and flows sensibly, but it does not need to be ``brilliant''. The Manual of Style does not need to be followed rigorously.
5.~The article contains supporting materials where appropriate. Illustrations are encouraged, though not required. Diagrams and an infobox etc. should be included where they are relevant and useful to the content. \\ \vspace{.5ex}
6.~The article presents its content in an appropriately understandable way. It is written with as broad an audience in mind as possible. Although Wikipedia is more than just a general encyclopedia, the article should not assume unnecessary technical background and technical terms should be explained or avoided where possible.
\end{tabular} \\ \hline
\end{tabular}
\label{tab:b_class_criteria}
\end{table*}
